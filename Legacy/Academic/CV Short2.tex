%%% CVS version control block - do not edit manually
%%%  $RCSfile$
%%%  $Revision$
%%%  $Date$
%%%  $Source$


% LaTeX resume using res.cls
\documentclass{res}
%\usepackage{helvetica} % uses helvetica postscript font (download helvetica.sty)
%\usepackage{newcent}   % uses new century schoolbook postscript font

\usepackage{eurofont}
\setlength{\topmargin}{-0.6in}  % Start text higher on the page 
\setlength{\textheight}{9.8in}  % increase textheight to fit more on a page
\setlength{\headsep}{0.2in}     % space between header and text
\setlength{\headheight}{12pt}   % make room for header
\usepackage{fancyhdr}  % use fancyhdr package to get 2-line header
\renewcommand{\headrulewidth}{0pt} % suppress line drawn by default by fancyhdr
\lhead{\hspace*{-\sectionwidth}Fiachra Matthews} % force lhead all the way left
\rhead{Page \thepage}  % put page number at right
\cfoot{}  % the footer is empty
\pagestyle{fancy} % set pagestyle for the document

\begin{document} 
\thispagestyle{empty} % this page does not have a header
\name{FIACHRA MATTHEWS}
\address{15 Thornhill Meadows\\
Celbridge, Co. Kildare\\
086 8044914}


\begin{resume}
\vspace{0.1in}
%\moveleft.5\sectionwidth\centerline{Objective: Position as Lecturer.}

\section{EDUCATION}
\vspace{0.1in}
 
    {\bf 2010} \underline{{\bf PhD}} Computer Science, NUI Maynooth, 
    (Awarded 2011).

    {\bf Integrated Real-Time Control And Processing Systems For
      Multi-Channel Near-Infrared Spectroscopy Based Brain Computer
      Interfaces}.  \\Supervisors: Prof. Barak Perlmutter \& Dr Tomas
    Ward \\ In this project I lead the development of a device that
    would enable disabled subects to control computer systems using thought
    alone. My major contributions revolved around creating specifications
    for the hardware and creating software algorithms for the control
    and processing of the data.   

 {\bf 2005} \underline{{\bf BSc}} Hons, 1$^{st}$ Class, Computer Science,
 N.U.I. Maynooth.

 This course included a six month industrial work placement with the
 IBM Microelectronics Division. It also included a final year project
 developing a software platform for a USB based inertial human motion
 tracking system which was awarded first prize in the Hewlett
 Packard Invent Award competition in 2005.

\section{Experience} \vspace{0.1in}
\vspace{0.1in}

    {\bf 2010 - Present Date}\\ Hamilton Institute, Post-Doctoral
    Fellow, Software Team Leader for Robocup team.
    \\ Duties include:

    \begin{itemize}
    \item Developing software systems (C++, Python) for the RoboEireann Standard
      Platform League (SPL) team.

    \item Developed software to exchange debug information with the
      robots via TCP in real time. (Linux and Android).

    \item Coordination and maintenance of the overall software project
      (Git).

    \item System administration for online bug-tracking system and Git
      hosting.  

    \item Supervisory roles for students on work placement and 10
      week internship projects (Summer Internship in Autonomous Robotics
      (SIAR)).

    \item Administrative duties include organizing and coordinating
      travel for the team to compete in International competitions.

\end{itemize}  

  
    {\bf 2005 (Summer)}\\Mercury Engineering, Sandyford Industrial Estate, Dublin.\\ 
    Visual Basic programming for the ``EMERGE'', SAP integration project.
    Duties included creating solutions for data migration and backup during the changeover period
    and consulting on document standards for compatibility with the new system.


    {\bf 2004 (March to September)}\\ Industrial work placement with
    IBM Microelectronics Division (IMD), Ireland.\\ Worked with the
    Process Engineering group collaborating on IT solutions to improve
    efficiency in the process of testing and validating of
    microchips. Created software in Lotus Script and VB for real-time
    test-floor progress report creation and display. Another project
    involved the creation of new bar code validation software written
    in JavaScript. This project also involved organising and managing
    its introduction into the process chain.

    {\bf 2002 \& 2003 (Summers)}\\Mercury Engineering, Intel, FAB 24
    project, Leixlip.  Document Control \& IT technician for the
    Life-Safety Systems Group. Duties included maintenance of electrical
    and structural drawings for the project, logging test results for the
    fire/smoke detection systems and general office IT maintenance.

\section{Technical  Skills} 
{\bf Languages \& Software:} 
 \begin{itemize} 
\item {\bf Highly proficient:} C, C++, Python, Labview, VB, Matlab, Latex, Bash Scripting.
\item {\bf Proficent:} x86 Assembly, OpenCV, OpenGL, LAMP server setup
  and maintanince.
\item {\bf Familiar:} Perl, JavaScript, PHP, MySQL.  
\end{itemize}

{\bf Operating Systems:} Windows, Android, Linux
                (mainly Ubuntu and Debian) and limited use of freeBSD.

\section{Teaching Experience}
\vspace{0.1in}

{\bf 2008} Lecturing, Department of Engineering, N.U.I. Maynooth,\\ 
EE620, Applied Computing for Engineers. 

{\bf 2008} Lecturing, Department of Engineering, N.U.I. Maynooth,\\ 
EE103, Computer Architechuire \& Digital Logic 

{\bf 2007} Lecturing, Department of Engineering, N.U.I. Maynooth,\\
Six lecture series in the theory and practice
of digital signal processing (DSP) for biomenical engeneering.

{\bf 2007} Lecturing, Department of Computer Science,
N.U.I. Maynooth,\\ 
Short, practical lecture series on an introduction to Labview.

{\bf 2005-2007} Demonstrating, Department of Computer Science, N.U.I. Maynooth,
\\ Undergraduate demonstration duties in:
\begin{itemize}
\item Computer Architechure
\item Digital Signal Processing
\end{itemize}
                


\section{Awards \& Achievements}
\vspace{0.1in}

{\bf 2005} Hewlett Packard Invent Award, Overall Winner.\\
This award recognised the best final year project in undergraduate
science and engineering programmes on the island of Ireland. Three
finalists were chosen from the submissions and the finalists had to
make a presentation to a panel of judges and answer questions on their
project. The panel of judges who decided on the winner comprised
people from both industry and academia.
    
{\bf 2006} Member of the George Mitchell Scholarship's Young Irish Leaders\\
The George J. Mitchell Scholarship is an American fellowship program
sponsored by the US-Ireland Alliance. It is designed to introduce and
connect generations of future American leaders to the island of
Ireland, while recognizing and fostering intellectual achievement,
leadership, and community. In 2006 the scholarship held its five year
reunion in Dublin. All previous scholars were invited as well as an
equal number of young Irish leaders including myself. We were brought
together as a group to discuss major issues and areas of shared
interest and how we could work together in the areas of Culture,
Politics, Business, Science and Medicine. In 2008 I participated in
the forum held to mark the 10th anniversary of the Good Friday
Agreement. Contributors to this forum included Taoiseach Bertie Ahern,
Gerry Adams, John Hume and General John De Chastelain. To date I
continue as member of the Young Irish Leaders.


\section{PUBLICATIONS}
\vspace{0.1in}

\subsection{\small{JOURNAL PUBLICATIONS}}

{C. ~Soraghan, C. ~Markham, \underline{F. ~Matthews},  \&  T. ~Ward,
  ``Triple wavelength led driver for optical brain--computer interfaces.'' \emph{
  Electronics Letters\/}, vol.~45, no.~8, pp. 392--394. 2009.}

{ \underline{F. ~Matthews}, B.~Pearlmutter, T.~Ward, C.~Soraghan, and C.~Markham,
  ``Hemodynamics for brain-computer interfaces,'' \emph{Signal Processing
  Magazine, IEEE}, vol.~25, no.~1, pp. 87--94, 2008.}


{ T.~E. Ward, C.~J. Soraghan, \underline{F. ~Matthews}, and C.~M. Markham, ``A concept for
  extending the applicability of constraint induced movement therapy through
  motor cortex activity feedback using a neural prosthesis,''
  \emph{Computational Intelligence and Neuroscience}, 2007.}
  
\subsection{\small{CONFERENCE PUBLICATIONS}}
\underline{F. ~Matthews}, C.~Soraghan, T.~E. Ward, C.~Markham and B.~A. Pearlmutter
   ``Software platform for rapid prototyping of NIRS brain-computer interfacing techniques''
   \emph{30th Annual International Conference Of The IEEE Engineering In Medicine And Biology Society}, Vancouver,
   Canada, 2008.

C.~Soraghan, \underline{F. ~Matthews}, C.~Markham, B.~A. Pearlmutter, R.~O�Neill, T.~E. Ward
   ''A 12-Channel, real-time near-infrared spectroscopy instrument for brain-computer interface applications''
   \emph{30th Annual International Conference Of The IEEE Engineering In Medicine And Biology Society}, Vancouver,
   Canada, 2008.

C.~J. Soraghan, T.~E. Ward, \underline{F. ~Matthews}, C.~Markham
   ''Optical Safety Assessment of a Near-Infrared Brain-Computer Interface''
   \emph{16th IET Irish Signals and Systems Conference} Galway, Ireland, 2008.

C. ~Soraghan, \underline{F. ~Matthews}, C. `Markham, B. ~A. Pearlmutter, and T. ~E. Ward
   ``Biophotonic Methods for Brain-Computer Interfaces''  (2007)
   \emph{Photonics Ireland}, Galway, Ireland

C.~Soraghan, \underline{F. ~Matthews}, D.~Kelly, T.~Ward, C.~Markham, B.~Pearlmutter, and
  R.~O'Neill, ``A dual-channel optical brain-computer interface in a gaming
  environment,'' in \emph{Proceedings of the 9th International Conference on
  Computer Games: AI, Animation, Mobile, Educational and Serious Games}, Dublin
  Institute of Technology, Ireland, Nov. 2006.

J. ~Foody, \underline{F. ~Matthews}, D. ~Kumar, C. ~Markham, T. ~Ward and B. ~Caulfield,
   ``A USB Interfaced Motion Capture Sensor, Using Tri-Axis Magnetic/Inertial Sensors For Use In Kinematic Studies''
   \emph{3rd European Medical and Biological Engineering Conference EMBEC'05 and IFMBE European Conference on Biomedical
   Engineering}, Prague, Czech Republic, 2005


\section{Referees}
\vspace{0.1in}

\begin{tabular}{| l | c | c |}
\hline
    & & \\  
  Prof. Richard Middleton & Hamilton Institute, N.U.I. Maynooth & richard.middleton@nuim.ie\\
   & & \\
  Prof. Barak Pearlmutter & Hamilton Institute, N.U.I. Maynooth & barak@cs.nuim.ie \\
   & & \\
  Dr. Tomas Ward & Dept. of Engineering, N.U.I. Maynooth & tomas.ward@nuim.ie\\
   & & \\

\hline
\end{tabular}


\end{resume}
\end{document}












