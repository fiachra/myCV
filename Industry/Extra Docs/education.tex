\section{Education}
\cventry{2005--2010}{PhD, Biomedical Engineering}{Hamilton Institute, NUI Maynooth}{}{}{}
\cvitem{Title}{\emph{Integrated Real-Time Control And Processing Systems For Multi-Channel Near-Infrared Spectroscopy Based Brain Computer Interfaces.}}
\cvitem{Supervisors}{Prof. Barak Perlmutter \& Dr Tomas Ward}
\cvitem{Description}{In this project I led the development of a device that would enable disabled subjects to control computer systems using thought alone. My major contributions revolved around creating specifications for the hardware and creating software algorithms for the control and processing of the data. Software was development using Matlab, Labview, C, C++ and Python.}

\cventry{2000--2005}{BSc, Computer Science \& Software Engineering}{NUI Maynooth, \textit{1st Hons}}{\newline{} Consists of a comprehensive computer science curriculum including modules in software engineering, databasing, networking, digital architecture and more. It also includes a six month industrial work placement module and major final year project}{}{}  % arguments 3 to 6 can be left empty

\cvitem{Work Placement}{Placement with IBM Microelectronics Division (IMD), Damastown Industrial Estate, Co. Dublin. I worked in the Process Engineering group collaborating on IT solutions to improve efficiency in the process of testing and validating of microchips. I wrote software for real-time test-floor progress report creation and display and created a new bar code validation software and organised and managed its introduction into the process chain.}
\cvitem{Final Year Project}{Development of a USB based goniometer for use as a platform for an inertial human motion tracking node.Supervised by Dr. Charles Markham, this project described an inertial orientation sensor based on solid-state technology, to be integrated into a larger system used for human motion tracking. Firmware development for data acquisition and USB communication was done in Microchip Assembly while the client driver and visualisation systems were developed using C, C++ and the openGL libraries. This project won first prize in the Hewlett Packard Invent Award in 2005.}