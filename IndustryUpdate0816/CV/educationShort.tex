\section{Education}
\cventry{2005--2010}{PhD, Biomedical Engineering}{Hamilton Institute, NUI Maynooth}{}{}{}
\cvitem{Title}{\emph{Integrated Real-Time Control And Processing Systems For Multi-Channel Near-Infrared Spectroscopy Based Brain Computer Interfaces.}}
\cvitem{Supervisors}{Prof. Barak Perlmutter \& Dr Tomas Ward}
\cvitem{Description}{In this project I led the development of a device that would enable disabled subjects to control computer systems using thought alone. The project involved designing specifications for the hardware and creating and implementing software algorithms for the control and processing of the data and user interaction. Software was development using Matlab, Labview, C, C++ and Python.}

\cventry{2000--2005}{BSc, Computer Science \& Software Engineering}{NUI Maynooth, \textit{1st Hons}}{\newline{} A comprehensive computer science curriculum including modules in software engineering, networking, digital architecture and more. Includes a six month industrial work placement and major final year project}{}{}  % arguments 3 to 6 can be left empty

\cvitem{Work Placement}{IBM Microelectronics Division (IMD), Damastown, Co. Dublin. Wrote software in VB, Lotous Script and Javascript with the Process Engineering group to improve efficiency in the testing and validating of microchips. }

\cvitem{Final Year Project}{Development of a USB based orientation sensor for use as a platform for a human motion tracking node. Device firmware was developed in Microchip Assembly while the client driver and front-end were developed using C and C++. This project won first prize in the Hewlett Packard Invent Award in 2005.}