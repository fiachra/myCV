%%% CVS version control block - do not edit manually
%%%  $RCSfile$
%%%  $Revision$
%%%  $Date$
%%%  $Source$


% LaTeX resume using res.cls
\documentclass{res}
%\usepackage{helvetica} % uses helvetica postscript font (download helvetica.sty)
%\usepackage{newcent}   % uses new century schoolbook postscript font

\usepackage{eurofont}
\setlength{\topmargin}{-0.6in}  % Start text higher on the page 
\setlength{\textheight}{9.8in}  % increase textheight to fit more on a page
\setlength{\headsep}{0.2in}     % space between header and text
\setlength{\headheight}{12pt}   % make room for header
\usepackage{fancyhdr}  % use fancyhdr package to get 2-line header
\renewcommand{\headrulewidth}{0pt} % suppress line drawn by default by fancyhdr
\lhead{\hspace*{-\sectionwidth}Fiachra Matthews} % force lhead all the way left
\rhead{Page \thepage}  % put page number at right
\cfoot{}  % the footer is empty
\pagestyle{fancy} % set pagestyle for the document

\begin{document} 
\thispagestyle{empty} % this page does not have a header
\name{FIACHRA MATTHEWS}
\address{6 Beaufield Grove\\
Maynooth, Co. Kildare\\
086 8044914}


\begin{resume}
\vspace{0.1in}
%\moveleft.5\sectionwidth\centerline{Objective: Position as Lecturer.}

\section{EDUCATION}
\vspace{0.1in}
 
    {\bf Ph.D.}, Hamilton Institute, NUI Maynooth, Computer Science, (Due for submission May, 2009)

    {\bf Integrated Real-Time Control And Processing Systems For
    Multi-Channel Near-Infrared Spectroscopy Based Brain Computer
    Interfaces}.  \\Supervisors: Prof. Barak Pearlmutter \& Dr. Tomas
    Ward \\The project outlines the control and processing of an
    optical brain computer interface (BCI). It contributes new
    applications of signal processing and activation classification
    methods to the area of Near-infrared spectroscopy based BCIs.
    Some of the skills I acquired necessary to this project included:
    \begin{itemize}
    \item Data processing and analysis
    \item Advanced signal processing methods
    \item Hardware synchronization
    \item Real-time systems development
    \end{itemize}

    Solutions were build in Matlab and C/C++ and a great deal of the
    interfacing was done with Labview. The project also involved
    spending some time researching and performing fMRI experiments in
    Rutgers University, New Jersey. Funded by SFI, this
    interdisciplinary project was run between the Dept. of Computer
    Science, Experimental Physics, Engineering and the Hamilton
    Institute. This project has so far 2 conference papers with 3 more
    in review and 2 journal papers including one in the IEEE Signal
    Processing Magazine.

    {\bf Primary Degree},  National University of Ireland, Maynooth, B.Sc.(Hons, 1$^{st}$ Class), 2005
    
    {\bf Computer Science and Software Engineering.}  \\A
    comprehensive computer science curriculum has been developed for
    this course based on the recommendations on computing curricula of
    the Association of Computing Machinery (ACM) and the Institute of
    Electrical and Electronics Engineers (IEEE). It also includes a
    six month industrial work placement module and major final year
    project.

    {\bf Final Year Project}

    {\bf Development of a USB based goniometer for use as a platform
    for an inertial human motion tracking node}.\\ Supervisor:
    Dr. Charles Markham \\This project described an inertial
    goniometer based on solid-state technology, to be integrated into
    a larger system used for human motion tracking. The system was
    built around a micro-controller containing a USB interface. The
    project involved creating the API, application and user interfaces
    writing the firmware for the micro-controller and testing \&
    commissioning the hardware. This project won first prize in the
    Hewlett Packard Invent Award in 2005.


\section{Teaching Experience}
\vspace{0.1in}
    {\bf Lecturing,} \\
    \textit{Department of Electronic Engineering, NUI Maynooth}\\
    \begin{itemize}
    \item Applied Computing (Matlab,Labview, Simulink). MSc Biomed Eng, 2008.
    \item Digital Architecture. BEng, First Year, 2008.
    \item Introduction to Signal Processing. Short lecture series.MSc Biomed
    Eng, 2008.
    \end{itemize}
    \textit{Department of Computer Science, NUI Maynooth}\\ 
    \begin{itemize}
    \item Two lecture introduction to Labview, Digital Architecture, BSc in
    Computer Science and Software engineering (CSSE), 3rd year, (2007).
    \end{itemize}
    {\bf Demonstrating,}\\
\textit{Department of Computer Science, NUI Maynooth (2005-2008)}
    \begin{itemize}
    \item 1st \& 3rd Year Digital Architechure.
    \item 1st Year Programming (C++ \& Java). 
    \item 2nd Year Signal Processing.
    
\section{Industrial Employment}
\vspace{0.1in}

    {\bf Summer 2005,}\\Mercury Engineering, Sandyford Industrial
    Estate, Dublin.  Visual Basic programming for data migration to
    SAP system.

    {\bf March to September, 2004}\\ Industrial work placement, IBM
    Microelectronics Division (IMD), Damastown industrial estate,
    Co. Dublin. Developing IT solutions within Process Engineering Group.

    {\bf Summer 2002 \& 2003,}\\Mercury Engineering, Intel, FAB 24
    project, Leixlip.  Document Control \& IT technician for the
    Life-Safety Systems Group. 


\section{Awards \& Achievements}
\vspace{0.1in}

    {\bf Hewlett Packard Invent Award, Overall Winner, 2005}\\ 
    An award for final year projects in Ireland in Engineering and
    Computer science. 

    {\bf Member of the George Mitchell Scholarship's Young Irish
    Leaders}\\ Invited to join the Mitchell Scholars as a young Irish
    leader for the five year aniversary for the Scholarship(2005). Recently
    joined the scholars at the 10 year aniversary of the Good Friday
    agreement in Belfast. Scholars and leaders were invited to attend
    a meeting of those who wrote the agreement to hear their thoughts
    on the peace process.   
 

\section{INTERESTS}
\vspace{0.1in}

{\bf Research Interests}\\

 \begin{itemize}
 \item Biomedical Engineering
 \item Parallel and Distributed Computing
 \item Signals and Systems
 \item Robotics
 \item Data Analysis
 \item Hardware Interfacing
 \item Computer Architecture
 \end{itemize}
 

\section{PUBLICATIONS}
\vspace{0.1in}

\subsection{\small{JOURNAL PUBLICATIONS}}
{\bf \underline{F. ~Matthews}, B.~Pearlmutter, T.~Ward, C.~Soraghan, and C.~Markham,
  ``Hemodynamics for brain-computer interfaces,'' \emph{Signal Processing
  Magazine, IEEE}, vol.~25, no.~1, pp. 87--94, 2008.}


{\bf T.~E. Ward, C.~J. Soraghan, \underline{F. ~Matthews}, and C.~M. Markham, ``A concept for
  extending the applicability of constraint induced movement therapy through
  motor cortex activity feedback using a neural prosthesis,''
  \emph{Computational Intelligence and Neuroscience}, 2007.}

\subsection{\small{CONFERENCE PUBLICATIONS}}
\underline{F. ~Matthews}, C.~Soraghan, T.~E. Ward, C.~Markham and B.~A. Pearlmutter
   ``Software platform for rapid prototyping of NIRS brain-computer interfacing techniques''
   \emph{30th Annual International Conference Of The IEEE Engineering In Medicine And Biology Society}, Vancouver,
   Canada, 2008.

C.~Soraghan, \underline{F. ~Matthews}, C.~Markham, B.~A. Pearlmutter, R.~O�Neill, T.~E. Ward
   ''A 12-Channel, real-time near-infrared spectroscopy instrument for brain-computer interface applications''
   \emph{30th Annual International Conference Of The IEEE Engineering In Medicine And Biology Society}, Vancouver,
   Canada, 2008.

C.~J. Soraghan, T.~E. Ward, \underline{F. ~Matthews}, C.~Markham
   ''Optical Safety Assessment of a Near-Infrared Brain-Computer Interface''
   \emph{16th IET Irish Signals and Systems Conference} Galway, Ireland, 2008.

C. ~Soraghan, \underline{F. ~Matthews}, C. `Markham, B. ~A. Pearlmutter, and T. ~E. Ward
   ``Biophotonic Methods for Brain-Computer Interfaces''  (2007)
   \emph{Photonics Ireland}, Galway, Ireland

C.~Soraghan, \underline{F. ~Matthews}, D.~Kelly, T.~Ward, C.~Markham, B.~Pearlmutter, and
  R.~O'Neill, ``A dual-channel optical brain-computer interface in a gaming
  environment,'' in \emph{Proceedings of the 9th International Conference on
  Computer Games: AI, Animation, Mobile, Educational and Serious Games}, Dublin
  Institute of Technology, Ireland, Nov. 2006.

J. ~Foody, \underline{F. ~Matthews}, D. ~Kumar, C. ~Markham, T. ~Ward and B. ~Caulfield,
   ``A USB Interfaced Motion Capture Sensor, Using Tri-Axis Magnetic/Inertial Sensors For Use In Kinematic Studies''
   \emph{3rd European Medical and Biological Engineering Conference EMBEC'05 and IFMBE European Conference on Biomedical
   Engineering}, Prague, Czech Republic, 2005

%\subsection{\small{PENDING PUBLICATIONS}}
%\vspace{0.1in}


\section{Referees}
\vspace{0.1in}

\begin{tabular}{| l | c | c |}
\hline
   & & \\
  Prof. Barak Pearlmutter & Hamilton Institute, NUI Maynooth & barak@cs.nuim.ie \\
   & & \\
  Dr. Charles Markham & Dept. of Computer Science, NUI Maynooth & charles.markham@nuim.ie\\
   & & \\
  Dr. Tomas Ward & Dept. of Engineering, NUI Maynooth & tomas.ward@nuim.ie\\
   & & \\

\hline
\end{tabular}


\end{resume}
\end{document}

