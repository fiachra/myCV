%%% CVS version control block - do not edit manually
%%%  $RCSfile$
%%%  $Revision$
%%%  $Date$
%%%  $Source$


% LaTeX resume using res.cls
\documentclass{res}
%\usepackage{helvetica} % uses helvetica postscript font (download helvetica.sty)
%\usepackage{newcent}   % uses new century schoolbook postscript font

\usepackage{eurofont}
\setlength{\topmargin}{-0.6in}  % Start text higher on the page 
\setlength{\textheight}{9.8in}  % increase textheight to fit more on a page
\setlength{\headsep}{0.2in}     % space between header and text
\setlength{\headheight}{12pt}   % make room for header
\usepackage{fancyhdr}  % use fancyhdr package to get 2-line header
\renewcommand{\headrulewidth}{0pt} % suppress line drawn by default by fancyhdr
\lhead{\hspace*{-\sectionwidth}Fiachra Matthews} % force lhead all the way left
\rhead{Page \thepage}  % put page number at right
\cfoot{}  % the footer is empty
\pagestyle{fancy} % set pagestyle for the document

\begin{document} 
\thispagestyle{empty} % this page does not have a header
\name{FIACHRA MATTHEWS}
\address{15 Thornhill Meadows\\
Celbridge, Co. Kildare\\
086 8044914}


\begin{resume}
\vspace{0.1in}
\moveleft.5\sectionwidth\centerline{Objective: Position as Lecturer.}

\section{EDUCATION}
\vspace{0.1in}
 
    \underline{{\bf Ph.D.}} Hamilton Institute, N.U.I. Maynooth, Computer Science,
    (Defended 2011, awaiting Graduation)

    {\bf Integrated Real-Time Control And Processing Systems For Multi-Channel Near-Infrared Spectroscopy Based Brain Computer Interfaces}.
    \\Supervisors: Professor Barak Perlmutter \& Dr. Tomas Ward
    \\The project outlines the control and processing of an optical Brain Computer Interface (BCI). It contributes new
    applications of signal processing and activation classification methods to the area of Near-infrared spectroscopy
    based BCIs as well as detailing the software necessary to control such a system including, data management, optical 
    modulation and demodulation, experimental protocol design,  user
    interaction and feedback. 

    Solutions were build in Matlab and C/C++ and a great deal of the
    interfacing was done with Labview. The project also involved spending
    some time researching and performing fMRI experiments in Rutgers
    University, New Jersey. The project was funded by SFI and was an
    interdisciplinary calloboration between the Deptartments of Computer
    Science, Experimental Physics, Electronic Engineering and the Hamilton
    Institute.

    To date there are four conference papers and three journal papers
    including one in the IEEE Signal Processing Magazine.

    \underline{{\bf Primary Degree}},  National University of Ireland, Maynooth, B.Sc.(Hons, 1$^{st}$ Class), 2005
    
    {\bf Computer Science and Software Engineering.}
    \\A comprehensive computer science curriculum has been developed for this course based on the recommendations on
    computing curricula of the Association of Computing Machinery (ACM) and the Institute of Electrical and Electronics 
    Engineers (IEEE). The degree program included modules in:
    
    \begin{itemize}
    \item Mathematics
    \item Algorithms \& Data Structures
    \item Computer Architecture
    \item Parallel \& Distributed Systems
    \item Natural Language Processing
    \item Neurocomputation
    \item Numerical Computation
    \item Cryptography
    \item Robotics
    \end{itemize}

    It also included a six month industrial work placement module and
    a major final year project.



    \underline{{\bf Final Year Project}}

    {\bf Development of a USB based goniometer for use as a platform for an inertial human motion tracking node}.\\
    Supervisor: Dr. Charles Markham
    \\This project described an inertial goniometer based on solid-state technology, to be integrated into
    a larger system used for human motion tracking. The system was built around a microcontroller containing a USB
    interface. The project involved creating the API, application and user interfaces, writing the firmware for the
    microcontroller in addition testing \& commissioning the hardware.

    This project revieved first prize of a bursary in the Hewlett Packard
    Invent Award in 2005 which was shared between myself and the
    department of Computer Science. 

\section{Academic Employment} \vspace{0.1in}

    {\bf May 2009 - Present Date}\\ Hamilton Institute, Post-Doctoral
    Position, Software Team Leader for Robocup team.
    \\ Duties include:

    \begin{itemize}
    \item Developing software systems for the RoboEireann Standard
      Platform League (SPL) team.

    \item Coordination maintenance of the overall software project.

    \item Supervisory roles for students on work placement or 10
      week internship projects (Summer Internship in Autonomous Robotics
      (SIAR)).

    \item Administrative duties include organizing and coordinating
      travel for the team to compete in International competitions.

\end{itemize}  
     

\section{Teaching Experience}
\vspace{0.1in}

{\bf Lecturing,} Department of Engineering, N.U.I. Maynooth,
2008\\ A part of the Masters in Biomedical Engineering I development
and delivered course a new course entitled ``Applied Computung for
Engineers''. The aim of this module was to assist post-graduate
students in how to analyse and present data from a research
perspective. Students were taught how to use Matlab for data analysis,
Labview for hardware interfacing and Simulink for systems modelling.

{\bf Lecturing,} Department of Engineering, N.U.I. Maynooth, 2008\\ 
I lectured the Computer Architechuire \& Digital Logic for first
years. This course was an introduction to the underlying conepts to
computational hardware systems.

    {\bf Lecturing,} Department of Engineering, N.U.I. Maynooth, 2007\\
    I delivered a six lecture introduction to the theory and practice
    of digital signal processing (DSP) as part of the Masters in
    Biomenical Engeneering. This course was designed to help the student
    to quickly grasp fundamentals of DSP.  I also assisted the lecturer in
    setting suitable exam questions on this material.

    {\bf Lecturing,} Department of Computer Science, N.U.I. Maynooth,
    2007\\ I was invited to give two lectures to CS352 on an introduction
    to Labview. I was given the responsibility of introducing the Labview
    components into this course as well as helping to choose new interfacing
    equipment for the Hardware laboratroy in the Deptartment of Computer
    Science. I was responsible too for setting up this equipment and
    organising laboratory assignments as well as assisting in the setting
    of exam questions for this material.

    {\bf Demonstrating,} Department of Computer Science, N.U.I. Maynooth,
2005-2007\\ Here subjects demonstrated included second and third year
computer architecture, first year programming and signal
processing. My responsibilities included supervising exams,
taking attendances, correcting lab exams and monitoring students
progress. Class size varied from five to about fifty in the
programming laboratory. 
    

 
\section{Industrial Employment}
\vspace{0.1in}

    {\bf Summer 2005,}\\Mercury Engineering, Sandyford Industrial Estate, Dublin.\\ 
    Here I programmed in Visual Basic for the ``EMERGE'', SAP integration project.
    Duties included creating solutions for data migration and backup during the changeover period
    and consulting on document standards for compatibility with the new system.


    {\bf March to September, 2004}\\ Industrial work placement with
    IBM Microelectronics Division (IMD), Damastown Industrial Estate,
    Co. Dublin.\\ I worked in the Process Engineering group collaborating
    on IT solutions to improve efficiency in the process of testing and
    validating of microchips. I wrote software in Lotus Script and VB for
    real-time test-floor progress report creation and display and created
    a new bar code validation software written in JavaScript.I further
    organised and managed it's introduction into the process chain.

    {\bf Summer 2002 \& 2003,}\\Mercury Engineering, Intel, FAB 24
    project, Leixlip.  Document Control \& IT technician for the
    Life-Safety Systems Group. Duties included maintenance of electrical
    and structural drawings for the project, logging test results for the
    fire/smoke detection systems and general office IT maintenance.


\section{Awards \& Achievements}
\vspace{0.1in}

    {\bf Hewlett Packard Invent Award, Overall Winner, 2005}\\
    This award recognises the best final year projects in undergraduate science and engineering programmes
    on the island of Ireland. Three finalists were chosen from the submissions and the finalists had to make
    a presentation to a panel of judges and answer questions on their project. The panel of judges who decided 
    on the winner comprised people from both industry and academia.
    
    {\bf Member of the George Mitchell Scholarship's Young Irish Leaders}\\
    The George J. Mitchell Scholarship is an American fellowship
    program sponsored by the US-Ireland Alliance. It is designed to
    introduce and connect generations of future American leaders to
    the island of Ireland, while recognizing and fostering
    intellectual achievement, leadership, and community. In 2006 the
    scholarship held its five year reunion in Dublin. All previous
    scholars were invited as well as an equal number of young Irish
    leaders including myself. We were brought together as a group to
    discuss major issues and areas of shared interest and how we
    could work together in the areas of Culture, Politics, Business,
    Science and Medicine. In 2008 I participated in the forum held to
    mark the 10th anniversary of the Good Friday
    Agreement. Contributors to this forum included Taoiseach
    Bertie Ahern, Gerry Adams, John Hume and General John De
    Chastelain. To date I continue as member of the Young Irish
    Leaders.
    


\section{INTERNATIONAL COLLABORATION}
\vspace{0.1in}


{\bf Rutgers University, New Jersey}\\
   Traveled to New Jersey to perform fMRI experiments in conjunction with the Cognitive Science laboratory in Rutgers and the
   University of Medicine and Dentistry of New Jersey (UMDNJ). This visit allowed me to perform concurrent NIRS and 
   fMRI studies as well as validate my experimental protocols.

   {\bf Colorado State University, Colorado}\\
   In 2007 Professor Charles W. Anderson of Colorado State University
   spent a six month sabbatical in the Hamilton institute. The professor
   worked with me on EEG and NIRS BCI systems and we remain in contact on
   the project.

\section{INTERESTS}
\vspace{0.1in}

{\bf Research Interests}\\
My main research interests lie in hardware software interaction, data
acquisition, processing and analysis. I would also hope to expand my
research into aspects of bio-medical engineering and robotics. Other
areas of research of interest to me are:

 \begin{itemize}
 \item Parallel and Distributed Computing
 \item Signals and Systems
 \item Theoretical Computer Science
 \item Hardware Interfacing
 \item Computer Architecture
 \item Computer Vision
 \end{itemize}
 
 {\bf Extra-curricular Interests} In my undergraduate and postgraduate
 years I have been heavily involved in a number of societies on
 campus. With the Drama Society I have directed, stage-managed, and
 acted in over fifteen productions.  I was a committee member of the
 Drama Society, IT Society (MINDS) and the Juggling Club.

\section{PUBLICATIONS}
\vspace{0.1in}

\subsection{\small{JOURNAL PUBLICATIONS}}

{\bf C. ~Soraghan, C. ~Markham, \underline{F. ~Matthews},  \&  T. ~Ward,
  ``Triple wavelength led driver for optical brain--computer interfaces.'' \emph{
  Electronics Letters\/}, vol.~45, no.~8, pp. 392--394. 2009.}

{\bf \underline{F. ~Matthews}, B.~Pearlmutter, T.~Ward, C.~Soraghan, and C.~Markham,
  ``Hemodynamics for brain-computer interfaces,'' \emph{Signal Processing
  Magazine, IEEE}, vol.~25, no.~1, pp. 87--94, 2008.}


{\bf T.~E. Ward, C.~J. Soraghan, \underline{F. ~Matthews}, and C.~M. Markham, ``A concept for
  extending the applicability of constraint induced movement therapy through
  motor cortex activity feedback using a neural prosthesis,''
  \emph{Computational Intelligence and Neuroscience}, 2007.}
  
\subsection{\small{CONFERENCE PUBLICATIONS}}
\underline{F. ~Matthews}, C.~Soraghan, T.~E. Ward, C.~Markham and B.~A. Pearlmutter
   ``Software platform for rapid prototyping of NIRS brain-computer interfacing techniques''
   \emph{30th Annual International Conference Of The IEEE Engineering In Medicine And Biology Society}, Vancouver,
   Canada, 2008.

C.~Soraghan, \underline{F. ~Matthews}, C.~Markham, B.~A. Pearlmutter, R.~O�Neill, T.~E. Ward
   ''A 12-Channel, real-time near-infrared spectroscopy instrument for brain-computer interface applications''
   \emph{30th Annual International Conference Of The IEEE Engineering In Medicine And Biology Society}, Vancouver,
   Canada, 2008.

C.~J. Soraghan, T.~E. Ward, \underline{F. ~Matthews}, C.~Markham
   ''Optical Safety Assessment of a Near-Infrared Brain-Computer Interface''
   \emph{16th IET Irish Signals and Systems Conference} Galway, Ireland, 2008.

C. ~Soraghan, \underline{F. ~Matthews}, C. `Markham, B. ~A. Pearlmutter, and T. ~E. Ward
   ``Biophotonic Methods for Brain-Computer Interfaces''  (2007)
   \emph{Photonics Ireland}, Galway, Ireland

C.~Soraghan, \underline{F. ~Matthews}, D.~Kelly, T.~Ward, C.~Markham, B.~Pearlmutter, and
  R.~O'Neill, ``A dual-channel optical brain-computer interface in a gaming
  environment,'' in \emph{Proceedings of the 9th International Conference on
  Computer Games: AI, Animation, Mobile, Educational and Serious Games}, Dublin
  Institute of Technology, Ireland, Nov. 2006.

J. ~Foody, \underline{F. ~Matthews}, D. ~Kumar, C. ~Markham, T. ~Ward and B. ~Caulfield,
   ``A USB Interfaced Motion Capture Sensor, Using Tri-Axis Magnetic/Inertial Sensors For Use In Kinematic Studies''
   \emph{3rd European Medical and Biological Engineering Conference EMBEC'05 and IFMBE European Conference on Biomedical
   Engineering}, Prague, Czech Republic, 2005


\section{Referees}
\vspace{0.1in}

\begin{tabular}{| l | c | c |}
\hline
    & & \\  
  Prof. Richard Middleton & Hamilton Institute, N.U.I. Maynooth & richard.middleton@nuim.ie\\
   & & \\
  Prof. Barak Pearlmutter & Hamilton Institute, N.U.I. Maynooth & barak@cs.nuim.ie \\
   & & \\
  Dr. Tomas Ward & Dept. of Engineering, N.U.I. Maynooth & tomas.ward@nuim.ie\\
   & & \\

\hline
\end{tabular}


\end{resume}
\end{document}












